% !TeX root = ../Rules.tex
% !TeX spellcheck = en_US
\section{The Official RoboCup Competition Rules}
\label{sec:robocup_rules}

This section contains rules that are not directly relevant for games and that may not apply at local opens.
However, these rules will be upheld at the yearly international RoboCup competition.

\subsection{Qualification Procedure and Code Usage}
\label{sec:qualification}

\subsection{Announcement of code and hardware usage}
\label{sec:code_and_hardware}

\subsection{Game Structure}
\label{sec:game_structure}

\subsection{Competition Mode}
\label{sec:competition_mode}

\subsection{Setup and Inspection}
\label{sec:setup_and_inspection}

\subsection{Competitions}
\label{sec:competitions}
%tbd: after division discussion

%\subsection{Champions Cup and Challenge Shield}
%\label{sec:champions_cup_challenge_shield}
%tbd: after division discussion

\subsection{Referee Duty and Selection}
\label{sec:referee_duty}

During pool play, the games will be refereed by members of teams from a different pool.

Each team has to referee a number of games.
A schedule will be released specifying the games for which each team is required to provide two referees.
Referees should report to the appropriate field at least ten minutes before the game is scheduled to start.

If a team fails to provide the due referees for a game in which they are scheduled to provide referees, it will be noted by the organizing committee and recorded as a \textbf{qualification penalty} (See section ???).

For each of the games, a team will be required either to provide the head referee and the operator of the GameController, or the assistant referees.
The two teams assigned to referee a game shall decide among themselves which roles each team will fulfill.
Note, however, that the head referee and the GameController should always be from the same team.

A team may swap their scheduled refereeing duties with another team, but the team listed on the referee schedule will be held accountable if referees fail to appear for a game they are scheduled to referee.

The requirement to referee may be an extreme hardship for extremely small teams.
If a team believes providing referees for games will be an extreme hardship, they must send an email explaining their situation to the Organizing Committee and Technical Committee at least two weeks before the first set up day of the competition.
The Organizing and Technical Committees will then consider the request and attempt to find an acceptable solution.

Referees must have good knowledge of the rules as applied in the tournament, and the operator of the GameController must be experienced in using that software.
Referees and the GameController should be selected among the more senior members of a team, and preferably have prior experience with games in RoboCup.

% If agreed upon by the referee teams or if scheduled by the Organizing Committee,
% volunteers may be allowed to serve as additional assistants.
% The extra assistants usually (but not necessarily) come from either of the teams
% scheduled for referee duty or from a pool of volunteers managed by the
% Organizing Committee.
% Details for people to register to such a pool are given just before the competition
% or at the competition site.

In each game, each of the teams playing shall be able to veto one and only one eligible referee with no reason required.
The veto must be delivered before the start of the \texttt{standby} phase or during a timeout.

The selection of referees, including the application of the above rules,
must always be made under the constraint that no member of a team may be allowed to
referee a game played by their own team under any circumstances.

\subsection{Rules for Forfeiting}
\label{sec:forfeit}

Teams who do not make a good faith effort to participate in a scheduled game are considered to forfeit the game.

If a team notifies the technical committee that they wish to forfeit less than two hours before their scheduled game time, simply fails to show up for their game, or decides during their game that they wish to forfeit, then the opposing team will play the match against an empty field.
However, any own goals will not be scored.
Hence, after an opponent forfeits, the team playing against an empty field cannot do worse than they were doing at the time the opponent decided to forfeit.
Teams may choose to forfeit at any stoppage of play.
However, once a forfeit is announced, they may not reverse this decision.

If a team notifies the technical committee that they wish to forfeit at least two hours before their schedule game time, the following procedure will be followed.
\begin{itemize}
  \item If a team chooses to forfeit a match in the round robin games the other team plays the match against an empty field.
    However, any own goals will not be scored.
  \item If a team chooses to forfeit in a knock-out game it gets replaced by the next best qualified team, \ie the team it kicked out or left behind in the round robins.
\end{itemize}

Note that there are a few unlikely cases that are not covered by these rules.
If a situation is not covered by these rules, the technical committee and the organizing committee will work together to make a decision.

Any forfeit will result in a qualification penalty being recorded (\cf \cref{sec:qualificationPenalties}) but the circumstances of the forfeit will affect the severity of the offense and the impact on future qualification.

\subsection{Source Code Releases}
\label{sec:code_release}

All teams that have participated in RoboCup must subsequently release code from that year's codebase.
The code must be licensed such that other RoboCup participants can use it, although the license may place conditions on its use.
The preferred type of release is the full source code of the software that was running in the team's last game at RoboCup.
In case this is not possible (\eg due to legal reasons), it is required that at least the source code related to the novel contributions (as given during the qualification process) is published.
Participation in technical challenges may come with additional requirements on the amount of components to be released.

The source code must be published and its availability announced on the league mailing list (\url{\LeagueEmail}) by \DTMdate{\CodeReleaseAnnouncementDate}.
Failing to publish source code by the deadline will result in a qualification penalty being recorded (\cf~\cref{sec:qualificationPenalties}).


\subsection{Subsequent Year Pre-qualification Procedure}
\label{sec:pre_qualification}

\subsection{Qualification Penalties}
\label{sec:qualification_penalties}

\subsection{Disqualification During Competition}
\label{sec:disqualification_during_competition}

A team may be disqualified during the RoboCup competition for:
\begin{itemize}
  \item A serious violation of the terms of a team's qualification
  \item Gaining a Qualification Penalty during the course of the competition~(\cf \cref{sec:qualificationPenalties})
  \item A serious breach of ethics, or serious behavior unbecoming of participants of RoboCup.
\end{itemize}

\textbf{Example.} A team promises to use their novel contribution in RoboCup games, but fails to do so.
Alternatively, a team deliberately misleads the technical committee about the novelty of their work and/or their contribution to the league, such that they are deemed to have copied another team.

A team can \textit{only} be disqualified by a decision of the \textit{Board of Trustees of the RoboCup Federation}.
The RoboCup Soccer League executive must petition the board in writing at their soonest possible availability.
The executive must simultaneously inform the relevant team of the petition in writing.

A disqualified team automatically forfeits all games~(\cf \cref{sec:forfeit}).
For practicality, the disqualification should not apply \textit{retroactively}.
However, by majority vote of the team leaders, provisions for retroactive disqualification may be made in the fairness of the affected teams.

\subsection{Awards}
\label{awards}

\subsubsection{Best Referee Voting}

Teams will be asked to vote for the team who has provided the best referrees throughout the competition.
Specifics on the voting process and awards will be detailed by the Organizing Committee during a referree meeting before
the start of the competition.

\subsubsection{Best Humanoid Award}

TBA

\subsection{Trophies}
\label{trophies}

