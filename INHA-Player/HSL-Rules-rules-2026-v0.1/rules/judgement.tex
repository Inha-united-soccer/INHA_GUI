% !TeX root = ../Rules.tex
% !TeX spellcheck = en_US

\law{5}{Human Officials}
\label{sec:referees}

Each match is run by a number of human officials. They are distinguished in two categories: referees, neutral persons who have the the duty to enforce the rules of play and ensure the game proceeds smoothly (including the GameController operator); and team officials, who belong to either of the teams and perform specific tasks for that team.

The referees are assigned before the game: example modalities are by scheduling in official games or by common agreement in friendly games.

As for team officials, each team selects theirs officials before the start of the game and communicates their selection to the head referee during the pre-game meeting. One person may cover multiple roles, unless otherwise specified.

\sublaw{Head Referee}
\label{sec:head_referee}

The head referee is the principal authority of the game and takes all decisions regarding the flow of the match and the enforcement of the rules.

\paragraph{Decisions of the Head Referee}
The head referee takes decisions to the best of their ability according to the rules and the spirit of the game. Any decision of the head referee is valid.

The head referee's decision is final and can not be changed afterwards, even by video proof. There is no discussion about decisions during the game, neither between the assistant referees and the head referee, nor between the audience or the teams and the head referee.

The head referee communicates their decision via verbal calls, occasionally complemented by a whistle sound.

\textbf{Calls.} The head referee announces decisions by a clear loud call, and (as required) whistle sound. The whistle, or where there is no whistle the first verbal word of the referees calls, defines the point in time at which the decision is made. The referees should make efforts to use consistent and clear calls, and it is preferable for referees to use the calls as specified in these rules.\footnote{The calls specified in these rules are detailed in English. With the agreement of the teams, the referees may use suitable calls in any language. The exception to this are technical challenges that depend on the calls as specified.} The intention of specifying the referee calls is for clarity and consistency across games.

\textbf{Whistle.} Where a whistle is required, the head referee first whistles and then announces the reason for the whistle. The head referee may choose to use any normal sports whistle. Each whistle sound should be short and not too loud as to interfere with other fields and simultaneous games. The head referee must only sound the whistle in circumstances described in these rules.

\paragraph{Powers and Duties}
\label{sec:referee_duty}

The head referee enforces the rules of the game and controls the match in cooperation with the assistant referees and the GameController operator.

The head referee adjudicates all infringements of the rules, and stops, suspends, or abandons the match accordingly. They indicate the restart of play after it has been stopped.

\improvement{This is the "advantage rule", decide if we want it or not} The head referee may allow play to continue when the team against which an offense has been committed will benefit from such an advantage, and penalizes the original offense if the anticipated advantage does not actually ensue.

When a robot commits multiple offenses at the same time, only the most serious one is called.

The head referee should act on the advice of the assistant referee regarding incidents that they has not seen.

The head referee has the right to take action against team officials who fail to conduct themselves in a responsible manner and may, at their discretion, expel them from the field of play and its immediate surrounds.

The head referee must stop the game if a human is seriously injured during play until that person has left the field.

The head referee can stop, suspend, or abandon the match because of outside interference of any kind, such as but not limited to: poor lighting or network conditions, or extraneous objects, animals or people entering the field.

The head referee can request any robot be stopped if it becomes a threat to the safety of objects, participants or spectators.

The head referee ensures that no unauthorized persons enter the field of play.

The head referee should avoid handling the ball (except for placing a ball for kick-off), and avoid handling the robots. Their duty is to monitor and adjudicate the game. The head referee should only handle robots and the ball if absolutely necessary to expedite game-play, where the robot handlers are otherwise occupied or too far away, and only if it doesn't compromise their own safety.

The head referee may decide at any point before or during a game to relocate any objects around the field, or direct persons to another position around the field.



\sublaw{Assistant Referees}
\label{sec:assistant_referee}

The assistant referees assist the head referee with offenses and other events when they have a clearer view. For example, they indicate when the ball has completely crossed over the line of the field, and which team is awarded the resulting goal or free kick. They can also point out rule violations that occurred outside of the head referee's view.

The assistant referees assist the head referee in officiating the match. They help monitor play, track the ball going out of bounds, and communicate with the head referee about incidents they observe.

Assistant referees may replace the ball when it goes off the field or becomes stuck between a player's feet. Robot handling is performed by the designated robot handlers (\cf~\Cref{sec:robot_handlers}), not by assistant referees, unless otherwise discussed with teams beforehand.

Assistant referees should only enter the field to execute a decision made by the head referee. They should not prevent robots from falling during the game.

A game normally has one or two assistant referees. The exact number is decided by the organizers of the competition.

\sublaw{GameController Operator}
\label{sec:gamecontroller_operator}

The operator of the GameController is a referee who sits at a PC in the technical area. As with the head referee, the operator should make efforts to use consistent and clear calls.

They will signal any change in the game state or penalties to the robots via the wireless as they are announced by the head referee. Note that for both kick-offs and goals, the moment of whistling is determining, not the verbal announcement of the head referee. They should repeat the call of the head referee as they do so, to confirm it was heard correctly.

The operator will also inform the robot handlers when a timed penalty is over and a robot has to be placed back on the field. They should announce events that occur automatically in the GameController due to elapsed time, such as the ball coming into play after a kick-off, penalty kick or free kick, or the state changing from \texttt{ready} to \texttt{set}.

They are also responsible for keeping the time of each half. They should count aloud the remaining seconds in a half once the time remaining is \qty{5}{\second} or less.

\sublaw{Equipment of the Referees}

All referees should wear a suitable referee jersey, and black or dark-blue socks, and they must avoid reserved colors (white and green) in their clothing.

The head referee must also be equipped with a whistle, a coin, and yellow and red cards.

The referees may also have equipment to communicate with each other, such as a headset or flags, if available.

\sublaw{Referee List for Friendly Games}
\label{sec:referee_list}

During a competition, especially (but not only) during the setup days, several teams may want to participate in friendly games with each other if a field is available to play in. People willing to volunteer for judging these games as head referee, assistant referee or GameController operator may submit their name to a list managed by the Organizing Committee, so that the teams organizing the friendly game are aware of their availability. This is especially recommended for those who wish to gain referee experience.

Ultimately, the teams organizing the friendly game are still free to decide whether to call volunteers from the list or otherwise choose their referees.

The Organizing Committee is in charge of maintaining the referee list and should be approached at the competition site if one should want to volunteer.

\sublaw{Team Captain}

The captain is a team official. They are the representative of the team for the game. They do not have to be the team leader.

The team captain represents the team during the pre-game meeting.

The team captain is the only person allowed to communicate with the referees. This includes making requests for pickup (Section \ref{sec:request_for_pick_up}).

\sublaw{Robot Handlers}
\label{sec:robot_handlers}

The robot handlers are team officials. They are the only people allowed to touch robots of their own teams (except for those that are picked up). Each team must provide one or two handlers.

During the game, the robot handlers stay in a designated area and must receive permission from the head referee before entering the field.

The robot handlers are tasked with removing their robots from the field when they are penalized, picked up, or otherwise removed from play. They must do so as swiftly as possible when instructed by the head referee, and with minimum interference with the game.

\improvement{Put either here or in forbidden actions: refusing to follow head ref's instructions is considered unsporting conduct}

When play is stopped, the robot handlers inform the head referee when their robots are ready to resume.

Robot handlers should only enter the field to execute a decision made by the head referee. They should not prevent robots from falling during an official game.

A robot handler may not touch a robot from a different team.

\sublaw{Humans Allowed on the Field}

During the game, starting at the Ready state, only the following people are allowed on the carpeted area (i.e. on the field and the surrounding border area):

\begin{itemize}
    \item The referees;
    \item The robot handlers;
    \item And no one else.
\end{itemize}

In addition, only the team captains and at most two more people per team are allowed next to the \improvement{The exact name of this area depends on how the tables are laid out} GameController tables. The remaining humans locate themselves around the other sides of the field if they want to watch the match. This allows the referees easier communication with the team and the GameController operator gets less disturbed.


\law{6}{Judgement}
\label{sec:judgement}

\sublaw{Pre-game Referee Meeting and Task Delegation}
\label{sec:referee_delegation}

At least \qty{15}{\minute} before the start of the game, the people who are going to serve as referees meet up to discuss the upcoming game. At least, they must assign among themselves the roles of head referee and GameController operator, with the remaining people acting as assistant referees.

The head referee should also communicate whether they are going to delegate any duties to the assistants. Other topics that ensure a smooth cooperation among the referees can also be discussed.

The head referee should talk to the assistants to determine what tasks or lesser decisions, if any, they wish to delegate to them to ensure that the game is arbitrated as smoothly as possible. This is left to the discretion of the head referee, based on their expertise in the role and their ability to focus on multiple events happening in the game at the same time and apply the corresponding rules.

The head referee must clearly communicate what tasks are delegated to which person, so that everyone understands their duties during the game.

If no agreement can be found, the default is that the responsibility for most calls and decisions falls upon the head referee, as determined by the rules.

Common examples of tasks that can be delegated are:

\begin{itemize}
    \item Indicating when the ball leaves the field
    \item Determining which team is to be given a free kick when the ball goes out of the field (Section~\ref{sec:kick_in}) and communicating this to the head referee
    \item Indicating violations requiring a penalty, especially if they happened out of the view of the head referee
    \item \todo{More examples can be added}
\end{itemize}

Note that the final decision is still made by the head referee.

The above list is not prescriptive: the head referee can always choose to delegate zero, one, some, or all of these tasks. It is also not exhaustive: through discussion among the head referee, the assistants, and the GameController operator, other tasks not listed here may be identified and delegated.

Care should be taken to not overburden any one person and not to blur the roles of head and assistant referees. Conflicts in the authority of the referees should be avoided, but if any do occur, the head referee's decision is final.

\sublaw{Pre-game Team Meeting}
Both teams send a representative called team captain to the field \qty{10}{\minute} before their match starts. This time should be used to welcome each other, assign team colors, choose side and kick-off, and discuss any other topics related to the match.

\sublaw{Referee -- Team Communication}
\label{sec:referee_team_communication}

During the match, only the team captains are allowed to communicate with the head referee.

After the match the teams thank the referees for their duty.

During all phases of the match teams and referees are communicating with respect to each other.

\sublaw{Referees During the Match}
\label{sec:referee_during_match}

The head referee and the assistant referees usually oversee the game from outside the field. Robot handlers enter the field to remove or place robots when penalties are applied (\cf~\Cref{sec:robot_handlers}). Referees should avoid interfering with the robots as much as possible.

\sublaw{Referees After the Match}

At the end of the game, the head referee must provide the Technical and Organizing Committees (or other relevant authorities) with a match report, which includes the final score and any relevant information on incidents that occurred before, during or after the match.
